\documentclass[]{article}
\usepackage{lmodern}
\usepackage{amssymb,amsmath}
\usepackage{ifxetex,ifluatex}
\usepackage{fixltx2e} % provides \textsubscript
\ifnum 0\ifxetex 1\fi\ifluatex 1\fi=0 % if pdftex
  \usepackage[T1]{fontenc}
  \usepackage[utf8]{inputenc}
\else % if luatex or xelatex
  \ifxetex
    \usepackage{mathspec}
  \else
    \usepackage{fontspec}
  \fi
  \defaultfontfeatures{Ligatures=TeX,Scale=MatchLowercase}
\fi
% use upquote if available, for straight quotes in verbatim environments
\IfFileExists{upquote.sty}{\usepackage{upquote}}{}
% use microtype if available
\IfFileExists{microtype.sty}{%
\usepackage{microtype}
\UseMicrotypeSet[protrusion]{basicmath} % disable protrusion for tt fonts
}{}
\usepackage[margin=1in]{geometry}
\usepackage{hyperref}
\hypersetup{unicode=true,
            pdftitle={Cluster Attractors},
            pdfborder={0 0 0},
            breaklinks=true}
\urlstyle{same}  % don't use monospace font for urls
\usepackage{color}
\usepackage{fancyvrb}
\newcommand{\VerbBar}{|}
\newcommand{\VERB}{\Verb[commandchars=\\\{\}]}
\DefineVerbatimEnvironment{Highlighting}{Verbatim}{commandchars=\\\{\}}
% Add ',fontsize=\small' for more characters per line
\usepackage{framed}
\definecolor{shadecolor}{RGB}{248,248,248}
\newenvironment{Shaded}{\begin{snugshade}}{\end{snugshade}}
\newcommand{\AlertTok}[1]{\textcolor[rgb]{0.94,0.16,0.16}{#1}}
\newcommand{\AnnotationTok}[1]{\textcolor[rgb]{0.56,0.35,0.01}{\textbf{\textit{#1}}}}
\newcommand{\AttributeTok}[1]{\textcolor[rgb]{0.77,0.63,0.00}{#1}}
\newcommand{\BaseNTok}[1]{\textcolor[rgb]{0.00,0.00,0.81}{#1}}
\newcommand{\BuiltInTok}[1]{#1}
\newcommand{\CharTok}[1]{\textcolor[rgb]{0.31,0.60,0.02}{#1}}
\newcommand{\CommentTok}[1]{\textcolor[rgb]{0.56,0.35,0.01}{\textit{#1}}}
\newcommand{\CommentVarTok}[1]{\textcolor[rgb]{0.56,0.35,0.01}{\textbf{\textit{#1}}}}
\newcommand{\ConstantTok}[1]{\textcolor[rgb]{0.00,0.00,0.00}{#1}}
\newcommand{\ControlFlowTok}[1]{\textcolor[rgb]{0.13,0.29,0.53}{\textbf{#1}}}
\newcommand{\DataTypeTok}[1]{\textcolor[rgb]{0.13,0.29,0.53}{#1}}
\newcommand{\DecValTok}[1]{\textcolor[rgb]{0.00,0.00,0.81}{#1}}
\newcommand{\DocumentationTok}[1]{\textcolor[rgb]{0.56,0.35,0.01}{\textbf{\textit{#1}}}}
\newcommand{\ErrorTok}[1]{\textcolor[rgb]{0.64,0.00,0.00}{\textbf{#1}}}
\newcommand{\ExtensionTok}[1]{#1}
\newcommand{\FloatTok}[1]{\textcolor[rgb]{0.00,0.00,0.81}{#1}}
\newcommand{\FunctionTok}[1]{\textcolor[rgb]{0.00,0.00,0.00}{#1}}
\newcommand{\ImportTok}[1]{#1}
\newcommand{\InformationTok}[1]{\textcolor[rgb]{0.56,0.35,0.01}{\textbf{\textit{#1}}}}
\newcommand{\KeywordTok}[1]{\textcolor[rgb]{0.13,0.29,0.53}{\textbf{#1}}}
\newcommand{\NormalTok}[1]{#1}
\newcommand{\OperatorTok}[1]{\textcolor[rgb]{0.81,0.36,0.00}{\textbf{#1}}}
\newcommand{\OtherTok}[1]{\textcolor[rgb]{0.56,0.35,0.01}{#1}}
\newcommand{\PreprocessorTok}[1]{\textcolor[rgb]{0.56,0.35,0.01}{\textit{#1}}}
\newcommand{\RegionMarkerTok}[1]{#1}
\newcommand{\SpecialCharTok}[1]{\textcolor[rgb]{0.00,0.00,0.00}{#1}}
\newcommand{\SpecialStringTok}[1]{\textcolor[rgb]{0.31,0.60,0.02}{#1}}
\newcommand{\StringTok}[1]{\textcolor[rgb]{0.31,0.60,0.02}{#1}}
\newcommand{\VariableTok}[1]{\textcolor[rgb]{0.00,0.00,0.00}{#1}}
\newcommand{\VerbatimStringTok}[1]{\textcolor[rgb]{0.31,0.60,0.02}{#1}}
\newcommand{\WarningTok}[1]{\textcolor[rgb]{0.56,0.35,0.01}{\textbf{\textit{#1}}}}
\usepackage{graphicx,grffile}
\makeatletter
\def\maxwidth{\ifdim\Gin@nat@width>\linewidth\linewidth\else\Gin@nat@width\fi}
\def\maxheight{\ifdim\Gin@nat@height>\textheight\textheight\else\Gin@nat@height\fi}
\makeatother
% Scale images if necessary, so that they will not overflow the page
% margins by default, and it is still possible to overwrite the defaults
% using explicit options in \includegraphics[width, height, ...]{}
\setkeys{Gin}{width=\maxwidth,height=\maxheight,keepaspectratio}
\IfFileExists{parskip.sty}{%
\usepackage{parskip}
}{% else
\setlength{\parindent}{0pt}
\setlength{\parskip}{6pt plus 2pt minus 1pt}
}
\setlength{\emergencystretch}{3em}  % prevent overfull lines
\providecommand{\tightlist}{%
  \setlength{\itemsep}{0pt}\setlength{\parskip}{0pt}}
\setcounter{secnumdepth}{0}
% Redefines (sub)paragraphs to behave more like sections
\ifx\paragraph\undefined\else
\let\oldparagraph\paragraph
\renewcommand{\paragraph}[1]{\oldparagraph{#1}\mbox{}}
\fi
\ifx\subparagraph\undefined\else
\let\oldsubparagraph\subparagraph
\renewcommand{\subparagraph}[1]{\oldsubparagraph{#1}\mbox{}}
\fi

%%% Use protect on footnotes to avoid problems with footnotes in titles
\let\rmarkdownfootnote\footnote%
\def\footnote{\protect\rmarkdownfootnote}

%%% Change title format to be more compact
\usepackage{titling}

% Create subtitle command for use in maketitle
\providecommand{\subtitle}[1]{
  \posttitle{
    \begin{center}\large#1\end{center}
    }
}

\setlength{\droptitle}{-2em}

  \title{Cluster Attractors}
    \pretitle{\vspace{\droptitle}\centering\huge}
  \posttitle{\par}
    \author{}
    \preauthor{}\postauthor{}
    \date{}
    \predate{}\postdate{}
  

\begin{document}
\maketitle

\hypertarget{rounded-raw-values-clustering}{%
\section{Rounded Raw Values
Clustering}\label{rounded-raw-values-clustering}}

\includegraphics{clustering_files/figure-latex/unnamed-chunk-7-1.pdf}
\includegraphics{clustering_files/figure-latex/unnamed-chunk-7-2.pdf}
\includegraphics{clustering_files/figure-latex/unnamed-chunk-7-3.pdf}
\includegraphics{clustering_files/figure-latex/unnamed-chunk-7-4.pdf}

\begin{verbatim}
## Warning: did not converge in 10 iterations
\end{verbatim}

\includegraphics{clustering_files/figure-latex/unnamed-chunk-7-5.pdf}
\includegraphics{clustering_files/figure-latex/unnamed-chunk-7-6.pdf}
\includegraphics{clustering_files/figure-latex/unnamed-chunk-7-7.pdf}

\hypertarget{discrete-values-clustering}{%
\section{Discrete Values Clustering}\label{discrete-values-clustering}}

\includegraphics{clustering_files/figure-latex/unnamed-chunk-8-1.pdf}

\begin{verbatim}
## Warning in arrows(0, 0, y[, 1L] * 0.8, y[, 2L] * 0.8, col = col[2L], length
## = arrow.len): zero-length arrow is of indeterminate angle and so skipped

## Warning in arrows(0, 0, y[, 1L] * 0.8, y[, 2L] * 0.8, col = col[2L], length
## = arrow.len): zero-length arrow is of indeterminate angle and so skipped

## Warning in arrows(0, 0, y[, 1L] * 0.8, y[, 2L] * 0.8, col = col[2L], length
## = arrow.len): zero-length arrow is of indeterminate angle and so skipped

## Warning in arrows(0, 0, y[, 1L] * 0.8, y[, 2L] * 0.8, col = col[2L], length
## = arrow.len): zero-length arrow is of indeterminate angle and so skipped

## Warning in arrows(0, 0, y[, 1L] * 0.8, y[, 2L] * 0.8, col = col[2L], length
## = arrow.len): zero-length arrow is of indeterminate angle and so skipped

## Warning in arrows(0, 0, y[, 1L] * 0.8, y[, 2L] * 0.8, col = col[2L], length
## = arrow.len): zero-length arrow is of indeterminate angle and so skipped

## Warning in arrows(0, 0, y[, 1L] * 0.8, y[, 2L] * 0.8, col = col[2L], length
## = arrow.len): zero-length arrow is of indeterminate angle and so skipped

## Warning in arrows(0, 0, y[, 1L] * 0.8, y[, 2L] * 0.8, col = col[2L], length
## = arrow.len): zero-length arrow is of indeterminate angle and so skipped
\end{verbatim}

\includegraphics{clustering_files/figure-latex/unnamed-chunk-8-2.pdf}
\includegraphics{clustering_files/figure-latex/unnamed-chunk-8-3.pdf}
\includegraphics{clustering_files/figure-latex/unnamed-chunk-8-4.pdf}
\includegraphics{clustering_files/figure-latex/unnamed-chunk-8-5.pdf}
\includegraphics{clustering_files/figure-latex/unnamed-chunk-8-6.pdf}
\includegraphics{clustering_files/figure-latex/unnamed-chunk-8-7.pdf}
\includegraphics{clustering_files/figure-latex/unnamed-chunk-8-8.pdf}

\#Raw DAC
\includegraphics{clustering_files/figure-latex/unnamed-chunk-10-1.pdf}
\includegraphics{clustering_files/figure-latex/unnamed-chunk-10-2.pdf}
\includegraphics{clustering_files/figure-latex/unnamed-chunk-10-3.pdf}
\includegraphics{clustering_files/figure-latex/unnamed-chunk-10-4.pdf}
\includegraphics{clustering_files/figure-latex/unnamed-chunk-10-5.pdf}
\includegraphics{clustering_files/figure-latex/unnamed-chunk-10-6.pdf}
\includegraphics{clustering_files/figure-latex/unnamed-chunk-10-7.pdf}
\includegraphics{clustering_files/figure-latex/unnamed-chunk-10-8.pdf}

\#discrete DAC
\includegraphics{clustering_files/figure-latex/unnamed-chunk-11-1.pdf}
\includegraphics{clustering_files/figure-latex/unnamed-chunk-11-2.pdf}
\includegraphics{clustering_files/figure-latex/unnamed-chunk-11-3.pdf}
\includegraphics{clustering_files/figure-latex/unnamed-chunk-11-4.pdf}
\includegraphics{clustering_files/figure-latex/unnamed-chunk-11-5.pdf}
\includegraphics{clustering_files/figure-latex/unnamed-chunk-11-6.pdf}
\includegraphics{clustering_files/figure-latex/unnamed-chunk-11-7.pdf}
\includegraphics{clustering_files/figure-latex/unnamed-chunk-11-8.pdf}

\#Both
\includegraphics{clustering_files/figure-latex/unnamed-chunk-12-1.pdf}
\includegraphics{clustering_files/figure-latex/unnamed-chunk-12-2.pdf}
\includegraphics{clustering_files/figure-latex/unnamed-chunk-12-3.pdf}
\includegraphics{clustering_files/figure-latex/unnamed-chunk-12-4.pdf}

\begin{verbatim}
## Warning: did not converge in 10 iterations
\end{verbatim}

\includegraphics{clustering_files/figure-latex/unnamed-chunk-12-5.pdf}
\includegraphics{clustering_files/figure-latex/unnamed-chunk-12-6.pdf}
\includegraphics{clustering_files/figure-latex/unnamed-chunk-12-7.pdf}

\begin{verbatim}
## Warning: Quick-TRANSfer stage steps exceeded maximum (= 50100)
\end{verbatim}

\includegraphics{clustering_files/figure-latex/unnamed-chunk-12-8.pdf}

\#Both discrete
\includegraphics{clustering_files/figure-latex/unnamed-chunk-13-1.pdf}

\begin{verbatim}
## Warning in arrows(0, 0, y[, 1L] * 0.8, y[, 2L] * 0.8, col = col[2L], length
## = arrow.len): zero-length arrow is of indeterminate angle and so skipped

## Warning in arrows(0, 0, y[, 1L] * 0.8, y[, 2L] * 0.8, col = col[2L], length
## = arrow.len): zero-length arrow is of indeterminate angle and so skipped

## Warning in arrows(0, 0, y[, 1L] * 0.8, y[, 2L] * 0.8, col = col[2L], length
## = arrow.len): zero-length arrow is of indeterminate angle and so skipped

## Warning in arrows(0, 0, y[, 1L] * 0.8, y[, 2L] * 0.8, col = col[2L], length
## = arrow.len): zero-length arrow is of indeterminate angle and so skipped

## Warning in arrows(0, 0, y[, 1L] * 0.8, y[, 2L] * 0.8, col = col[2L], length
## = arrow.len): zero-length arrow is of indeterminate angle and so skipped

## Warning in arrows(0, 0, y[, 1L] * 0.8, y[, 2L] * 0.8, col = col[2L], length
## = arrow.len): zero-length arrow is of indeterminate angle and so skipped

## Warning in arrows(0, 0, y[, 1L] * 0.8, y[, 2L] * 0.8, col = col[2L], length
## = arrow.len): zero-length arrow is of indeterminate angle and so skipped
\end{verbatim}

\includegraphics{clustering_files/figure-latex/unnamed-chunk-13-2.pdf}
\includegraphics{clustering_files/figure-latex/unnamed-chunk-13-3.pdf}
\includegraphics{clustering_files/figure-latex/unnamed-chunk-13-4.pdf}
\includegraphics{clustering_files/figure-latex/unnamed-chunk-13-5.pdf}
\includegraphics{clustering_files/figure-latex/unnamed-chunk-13-6.pdf}
\includegraphics{clustering_files/figure-latex/unnamed-chunk-13-7.pdf}
\includegraphics{clustering_files/figure-latex/unnamed-chunk-13-8.pdf}

\hypertarget{knn-for-discrete-both}{%
\section{Knn for discrete Both}\label{knn-for-discrete-both}}

\begin{Shaded}
\begin{Highlighting}[]
\NormalTok{df5 <-}\StringTok{ }\NormalTok{all_T}
\NormalTok{df5}\OperatorTok{$}\NormalTok{class <-}\StringTok{ }\KeywordTok{c}\NormalTok{(}\StringTok{"cancerous"}\NormalTok{, }\StringTok{"normal"}\NormalTok{, }\KeywordTok{rep}\NormalTok{(}\StringTok{"unknown"}\NormalTok{, }\DecValTok{1000}\NormalTok{))}
\NormalTok{kn <-}\StringTok{ }\KeywordTok{knn}\NormalTok{(}\DataTypeTok{train =}\NormalTok{ df5[}\DecValTok{1}\OperatorTok{:}\DecValTok{2}\NormalTok{,}\OperatorTok{-}\DecValTok{117}\NormalTok{], }\DataTypeTok{test =}\NormalTok{ df5[}\OperatorTok{-}\KeywordTok{c}\NormalTok{(}\DecValTok{1}\NormalTok{,}\DecValTok{2}\NormalTok{),}\OperatorTok{-}\DecValTok{117}\NormalTok{], }\DataTypeTok{cl =}\NormalTok{ df5[}\DecValTok{1}\OperatorTok{:}\DecValTok{2}\NormalTok{,]}\OperatorTok{$}\NormalTok{class, }\DataTypeTok{k =} \DecValTok{2}\NormalTok{)}
\NormalTok{res <-}\StringTok{ }\KeywordTok{data.frame}\NormalTok{(}\StringTok{"attractor"}\NormalTok{ =}\StringTok{ }\KeywordTok{row.names}\NormalTok{(df5)[}\OperatorTok{-}\KeywordTok{c}\NormalTok{(}\DecValTok{1}\NormalTok{,}\DecValTok{2}\NormalTok{)], }\StringTok{"cluster"}\NormalTok{=}\StringTok{ }\KeywordTok{as.character}\NormalTok{(kn), }\DataTypeTok{stringsAsFactors =} \OtherTok{FALSE}\NormalTok{)}
\end{Highlighting}
\end{Shaded}


\end{document}
